%% Matjaz Perc
%% César A Hidalgo
%% Vittoria Colizza
%% Alessandro Vespignani
%% Luís A. Nunes Amaral

To the Editors,

We are very pleased to submit our manuscript for consideration at PLoS
ONE:\\
\textbf{``Predicting Flow Reversals in a Computational Fluid Dynamics Simulated Thermosyphon using Data Assimilation
''.}

Our major contributions and findings are as follows:

\textbf{1.}
We build a general data assimilation framework for MATLAB and Julia which has two key advantages:\\
\mbox{}\ \textbf{a.} it utilizes an object-oriented design such that the model and data assimilation algorithm code are separate and\\
\mbox{}\ \textbf{b.} this allows incorporation of new models and DA techniques independently.

\textbf{2.}
We build and test a 2D and 3D computational model of our physical thermosyphon experiment with appropriate boundary conditions, meshing, and solving in OpenFOAM to access the experimentally seen dynamical regime of deterministic, nonperiodic flow.
This model of ``Chaos in an Atmosphere Hanging on a Wall'' facilitates our ability to test data assimilation and ensemble forecasting, with the ultimate goal of improving weather predictions.

\textbf{3.}
We augment the localized Ensemble Transform Kalman Filter by introducing an adaptive covariance localization scheme which is\\
\mbox{}\ \textbf{a.} sufficiently fast for implementation in large models and\\
\mbox{}\ \textbf{b.} improves flow reversal prediction skill particularly when spatial observation density is decreased.

\textbf{4.}
We apply Dynamic Mode Decomposition and find that modes correspond to a hidden, lower dimensional, system.
This indicates that DMD could be used to improve predictability of reversals, especially in situations where large systems can be approximated by with dimension reduction.\\

We look forward to hearing your decision.

Yours sincerely and on behalf of the manuscript's authors,


%% Professor
%% Director of the Vermont Complex Systems Center
%% Co-Director, Computational Story Lab
%% Vermont Advanced Computing Center
%% Department of Mathematics and Statistics
%% The University of Vermont
