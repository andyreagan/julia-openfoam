\begin{abstract}
A thermal convection loop is a circular chamber filled with water, heated on the bottom half and cooled on the top half.
With sufficiently large forcing of heat, the direction of fluid flow in the loop oscillates chaotically, forming an analog to the Earth's weather.
As is the case for state-of-the-art weather models, we only observe the statistics over a small region of state space, making prediction difficult.
To overcome this challenge, data assimilation methods, and specifically ensemble methods, use the computational model itself to estimate the uncertainty of the model to optimally combine these observations into an initial condition for predicting the future state.
First, we build and verify four distinct DA methods.
Then, a computational fluid dynamics simulation of the loop and a reduced order model are both used by these DA methods to predict flow reversals.
The results contribute to a testbed for algorithm development.
\end{abstract}
