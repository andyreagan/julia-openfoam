


\begin{figure*}[tp!]
    \begin{center}
      %\includegraphics[width=\textwidth]{calorimeterMaps.eps}
    \end{center}
    \caption{\textbf{
      Choropleth maps indicating 
      \textbf{(A)} caloric input $\calin$,
      \textbf{(B)} caloric output $\calout$,
      and
      \textbf{(C)} caloric balance $\calbal$
      in the contiguous United States (including the District of 
      Columbia) based on 50 million geotagged 
      tweets taken from 2011-2012.
}       Darker means higher values as per the color bars.
      The histograms in Figs.~\ref{fig:fluxwell.histograms},
      \ref{fig:fluxwell.histograms-food},
      and \ref{fig:fluxwell.histograms-activities} 
      show the specific rankings according to the three variables.
      The phrases in \textbf{(A)} and \textbf{(B)}
      are those whose increased usage contribute the most 
      to a population's $\calin$ and $\calout$ differing from the average
      (see Sec.~\ref{subsec:fluxwell.phraseshifts}).
    }
    \label{fig:fluxwell.maps}
\end{figure*}


\begin{figure}[tp!]
    \begin{center}
      %\includegraphics[width=\columnwidth]{stateScatter.eps}
    \end{center}
    \caption{\textbf{
      Plots for the contiguous US 
      showing the lack of correlation between
      caloric input $\calin$ and caloric output $\calout$,
      demonstrating their separate value
      as they bear different kinds of information.
}       The Pearson correlation coefficient 
      $\rhopearson$
      is -0.13
      and the best line of fit slope is $m=$ -1.69.
      Fig.~\ref{fig:fluxwell.scatterplots} adds
      plots of 
      $\calbal$ as a function of $\calin$ and $\calout$.
    }
    \label{fig:fluxwell.scatterplot_calin_calout}
\end{figure}


\begin{figure*}[tp!]
    \begin{center}
      %\includegraphics[width=\textwidth]{calorimeterBarplots_abbrevs.eps}
    \end{center}
    \caption{\textbf{
      Histograms of 
      caloric intake $\calin$ (food),
      caloric output $\calout$ (activity), 
      and
      caloric balance $\calbal$
      for the states of the contiguous US,
      all ranked by decreasing $\calbal$.
}       Bars indicate the difference in the three quantities from
      the overall average with
      colors corresponding to those used in Fig.~\ref{fig:fluxwell.maps}.
      We provide the same set of histograms re-sorted by
      $\calin$ and $\calout$
      in Figs.~\ref{fig:fluxwell.histograms-food}
      and~\ref{fig:fluxwell.histograms-activities}.
    }
    \label{fig:fluxwell.histograms}
\end{figure*}


\begin{figure*}[tbp!]
  \begin{center}
    %\includegraphics[width=\textwidth]{food-activity-phrase-shift003.eps} 
  \end{center}
    \caption{\textbf{
      Phrase shifts showing which food phrases 
      and physical activity phrases
      have the most influence on Colorado and Mississippi's 
      top and bottom ranking for caloric balance,
      when compared
      with the average for the contiguous United States.
}       Note that phrases are lemmas representing phrase categories.
      Overall, Colorado scores lower on Twitter food calories
      (257.4 versus 271.7)      
      and higher 
      on physical activity calories 
      (203.5 versus 161.3)
      than Mississippi.
      We provide interactive phrase shifts for 
      as part of the paper's online
      appendix at
      \url{http://compstorylab.org/share/papers/alajajian2015a/}
      and at
      \url{http://panometer.org/instruments/lexicocalorimeter}.
      We explain phrase (word) shifts in the main text
      (see Eqs.~\ref{eq:fluxwell.delta} and~\ref{eq:fluxwell.delta-final}),
      and in full depth in~\cite{dodds2011e} 
      and~\cite{dodds2015a}
      and online at
      \url{http://hedonometer.org}~\cite{wordshiftexplanations2014a}.
    }
  \label{fig:fluxwell.wordshiftexample}
\end{figure*}


\begin{figure*}[tp!]
    \begin{center}
      %\includegraphics[width=\textwidth]{scatterPlots.eps}
    \end{center}
    \caption{\textbf{
      Six demographic quantities compared with
      caloric balance $\calbal$ for the contiguous US.
}       The inset values
      are the Spearman correlation coefficient $\rhospearman$,
      and
      the Benjamini-Hochberg $q$-value.
      See Tab.~\ref{tab:fluxwell.corrTable} for
      a full summary of the 37 demographic quantities
      studied here.
    }
    \label{fig:fluxwell.demog-scatterplots}
\end{figure*}


\begin{figure*}[tp!]
    \begin{center}
      %\includegraphics[width=\textwidth]{2015-07-03lexicocalorimeter-dashboard.eps}
    \end{center}
    \caption{\textbf{
      Screenshot of the interactive dashboard for our prototype
      Lexicocalorimeter site (taken 2015/07/03).
}       An archived development version can be found
      as part of our paper's Online Appendices at
      \url{http://compstorylab.org/share/papers/alajajian2015a/maps.html},
      and a full dynamic implementation will be part of our
      Panometer project at
      \url{http://panometer.org/instruments/lexicocalorimeter}.
    }
    \label{fig:fluxwell.lexicocalorimeter}
\end{figure*}
\clearpage