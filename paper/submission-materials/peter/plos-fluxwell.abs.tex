We propose and develop a Lexicocalorimeter: 
an online, interactive instrument for measuring the ``caloric
content'' of social media and other large-scale texts.
We do so by constructing extensive yet improvable
tables of food and activity related phrases,
and respectively assigning them with sourced estimates of caloric 
intake and expenditure.
We show that for Twitter, our naive measures of 
``caloric input'', ``caloric output'', and
the ratio of these measures---``caloric balance''---are all strong
correlates with health and well-being demographics for the
contiguous United States.
Our caloric balance measure outperforms both its constituent quantities;
is tunable to specific demographic measures such as diabetes rates;
provides a real-time signal reflecting a population's health;
and has the potential to be used alongside
traditional survey data in the development 
of public policy and collective self-awareness.
Because our Lexicocalorimeter is a linear superposition
of principled phrase scores,
we also show we can move beyond correlations to explore what people talk about in 
collective detail, and
assist in the understanding and explanation of how population-scale conditions vary,
a capacity unavailable to black-box type methods.
