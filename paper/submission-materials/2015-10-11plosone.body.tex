%% Matjaz Perc
%% César A Hidalgo
%% Vittoria Colizza
%% Alessandro Vespignani
%% Luís A. Nunes Amaral

To the Editors,

We are very pleased to submit our manuscript for consideration at PLoS
ONE:\\
\textbf{``The Lexicocalorimeter:
Gauging public health through caloric input and output on social media''.}

Our major contributions and findings are as follows:

\textbf{1.}
We develop a highly novel and effective method for:\\
\mbox{}\ \textbf{a.} extracting food and activity
related phrases in any text, and \\
\mbox{}\ \textbf{b.} then converting
these phrases into estimates of caloric input and output.

\textbf{2.}
We study the contiguous US in depth through the
"caloric content'' of geotagged tweets, 
first by comparing 
the 48 states and DC with each other, and second by correlating our
caloric measures with a diverse set of 37 health, well-being, and demographic
quantities
such as obesity rates, brain health, and daily consumption of fruit.

\textbf{3.}
Our measure may be tuned to focus on specific health concerns
such as diabetes rates.

\textbf{4.}
Our approach is fast, transparent, linear, and improvable,
and, in great contrast with black box approaches, allows
us to show via our sophisticated ``phrase shifts''
why a population has a given caloric intake or outlay.
\textbf{
With the ultimate goal of enabling better informed public-health initiatives,
our Lexicocalorimeter affords both vast new descriptive power
and a potential for gaining new insight into profound problems
such as obesity.}

\textbf{5.}
We have fully implemented our new instrument online as part of
our broader Panometer project: \url{http://panometer.org/instruments/lexicocalorimeter}.
As of this submission, the site allows users to deeply explore and
share the rich patterns of food and activity
phrases across the contiguous US, and the resultant "caloric texture".

\textbf{We believe our Lexicocalorimeter will steadily evolve
into a more and more powerful instrument for monitoring and positively
affecting population-scale health.}

We look forward to hearing your decision.

Yours sincerely and on behalf of the manuscript's authors,


%% Professor
%% Director of the Vermont Complex Systems Center
%% Co-Director, Computational Story Lab
%% Vermont Advanced Computing Center
%% Department of Mathematics and Statistics
%% The University of Vermont
